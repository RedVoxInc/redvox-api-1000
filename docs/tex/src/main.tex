%! Author = RedVox, Inc
%! Date = 4/16/20

% Preamble
\documentclass[11pt]{article}

% Packages
\usepackage{amsmath}

% Document
\begin{document}

    \section{Sample Rate Equations}\label{sec:sample-rate-equations}
    All timestamps and sample intervals are recorded as microseconds.
    Sample rates are given in Hz (samples per second).

    \subsection{Converting timestamps into sample intervals}\label{subsec:converting-timestamps-into-sample-intervals}

    \begin{enumerate}
        \item Let $timestamps$ be an array of consecutive timestamps
        \item Let $diff()$ be a function that computes and returns the difference between consecutive items (e.g. $diff(1,3,9)=2,6$)
        \item Let $len()$ be a function that returns the total number of items in a collection
    \end{enumerate}

    Then, the mean ($\mu SI$) and standard deviation ($\sigma SI$) of the sample interval ($SI$) can be found with

    \begin{align}
        SI &= diff(timestamps) \\
        M &= len(SI) \\
        \mu SI &= \frac{\sum_{i=0}^{M-1} SI_i}{M} \\ \label{eq:mu_si}
        \sigma SI &= \sqrt{\frac{(\sum_{i=0}^{M-1} SI_i - \mu SI)^2}{M}}
    \end{align}

    \subsection{Converting sample intervals into sample rates}\label{subsec:converting-sample-intervals-into-sample-rates}

    He we will find the mean sample rate ($\mu SR$) in Hz and the standard deviation of the sample rate ($\sigma SR$) in Hz.

    If $\mu SI$ (eq~\ref{eq:mu_si}) $\leq 0$, then both $\mu SR$ and $\sigma SR$ should be set 0. Otherwise, these values can be found with the following equations:

    \begin{align}
        \mu SR &= \frac{1}{\mu SR * 10^{-6}} \\
        \sigma SR &\approx \mu SR^2 * \sigma SI * 10^{-6}
    \end{align}

\end{document}
